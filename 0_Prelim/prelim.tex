% This file contains all the necessary setup and commands to create
% the preliminary pages according to the buthesis.sty option.

\title{MakerFluidics: Low Cost Microfluidics for Synthetic Biology}

\author{Ryan Silva}

% Type of document prepared for this degree:
%   1 = Master of Science thesis,
%   2 = Doctor of Philisophy dissertation.
%   3 = Master of Science thesis and Doctor of Philisophy dissertation.
\degree=2

\prevdegrees{B.S., United States Air Force Academy, 2005\\
	M.S., The Air Force Institute of Technology, 2007}

\department{Department of Electrical and Computer Engineering}

% Degree year is the year the diploma is expected, and defense year is
% the year the dissertation is written up and defended. Often, these
% will be the same, except for January graduation, when your defense
% will be in the fall of year X, and your graduation will be in
% January of year X+1
\defenseyear{2017}
\degreeyear{2017}

% For each reader, specify appropriate label {First, Second, Third},
% then name, and title. IMPORTANT: The title should be:
%   "Professor of Electrical and Computer Engineering",
% or similar, but it MUST NOT be:
%   Professor, Department of Electrical and Computer Engineering"
% or you will be asked to reprint and get new signatures.
% Warning: If you have more than five readers you are out of luck,
% because it will overflow to a new page. You may try to put part of
% the title in with the name.
\reader{First}{Douglas M. Densmore, Ph.D.}{Associate Professor of Electrical and Computer Engineering \\ Associate Professor of Biomedical Engineering}
\reader{Second}{Jason W. Holder, Ph.D.}{Principal Member, Technical Staff \\ The Charles Stark Draper Laboratory, Inc.}
\reader{Third}{Ajay Joshi, Ph.D.}{Associate Professor of Electrical and Computer Engineering}
\reader{Fourth}{Michel Kinsy, Ph.D.}{Assistant Professor of Electrical and Computer Engineering}
\reader{Fifth}{James Galagan, Ph.D.}{Associate Professor of Biomedical Engineering \\ Boston University, College of Engineering \vspace{6pt}\\ Associate Professor of Microbiology \\ Boston University, School of Medicine}

% The Major Professor is the same as the first reader, but must be
% specified again for the abstract page. Up to 4 Major Professors
% (advisors) can be defined. 
\numadvisors=2
\majorprof{Douglas M. Densmore, Ph.D.}{{Associate Professor of Electrical and Computer Engineering \\ Associate Professor of Biomedical Engineering}}
%\majorprof{Douglas Densmore, Ph.D.}{{Associate Professor of Electrical and Computer Engineering}}
\majorprofb{Jason W. Holder, Ph.D.}{{Principal Member, Technical Staff \\ The Charles Stark Draper Laboratory, Inc.}}
%\majorprofb{Jason W. Holder, PhD}{{Principal Member of Technical Staff, The Charles Stark Draper Laboratory}}
%\majorprofc{First M. Last, PhD}{{Professor of Astronomy}}
%\majorprofd{First M. Last, PhD}{{Professor of Biomedical Engineering}}

%%%%%%%%%%%%%%%%%%%%%%%%%%%%%%%%%%%%%%%%%%%%%%%%%%%%%%%%%%%%%%%%  

%                       PRELIMINARY PAGES
% According to the BU guide the preliminary pages consist of:
% title, copyright (optional), approval,  acknowledgments (opt.),
% abstract, preface (opt.), Table of contents, List of tables (if
% any), List of illustrations (if any). The \tableofcontents,
% \listoffigures, and \listoftables commands can be used in the
% appropriate places. For other things like preface, do it manually
% with something like \newpage\section*{Preface}.

% This is an additional page to print a boxed-in title, author name and
% degree statement so that they are visible through the opening in BU
% covers used for reports. This makes a nicely bound copy. Uncomment only
% if you are printing a hardcopy for such covers. Leave commented out
% when producing PDF for library submission.
%\buecethesistitleboxpage

% Make the titlepage based on the above information.  If you need
% something special and can't use the standard form, you can specify
% the exact text of the titlepage yourself.  Put it in a titlepage
% environment and leave blank lines where you want vertical space.
% The spaces will be adjusted to fill the entire page.
\maketitle
\cleardoublepage

% The copyright page is blank except for the notice at the bottom. You
% must provide your name in capitals.
\copyrightpage
\cleardoublepage

% Now include the approval page based on the readers information
\approvalpage
\cleardoublepage

% Here goes your favorite quote. This page is optional.
\newpage
%\thispagestyle{empty}
\phantom{.}
\vspace{4in}

\begin{singlespace}
\begin{quote}
  \textit{Shout out to Jesus}
\end{quote}
\end{singlespace}

% \vspace{0.7in}
%
% \noindent
% [The descent to Avernus is easy; the gate of Pluto stands open night
% and day; but to retrace one's steps and return to the upper air, that
% is the toil, that the difficulty.]

\cleardoublepage

% The acknowledgment page should go here. Use something like
% \newpage\section*{Acknowledgments} followed by your text.
\newpage
\section*{\centerline{Acknowledgments}}
I would like to first thank my amazing, beautiful wife who puts up with my nonsense. Shout out to Baby Jay, even though he was a late arrival to this party. When I come home from a long day at the lab and see the smile on his little face, I just know he's about to jab me with something.

There is no better PhD advisor than Douglas Densmore. He taught me the one little word that will get me through academic life and beyond: \emph{tranquillo}. Like me, he understands that sometimes in life you only get to order one \emph{primi} [sic]. In all seriousness, I have yet to describe my PhD experience to anyone without inducing some serious PI envy. Thank you for everything and I will cherish the days you allowed me to spend in your lab. Thanks to the Fluigi crew (Josh, Krishna, Ali) and Prashant for always getting me home safe and on-time.

Special thanks to Jason Holder, who pushed me to develop a thesis of which I can be proud. Thank you for letting me be a part of your lab and your team. What can I say about the Acoustic Boyz? Thanks to Parker Dow for teaching me everything I know; thanks to Charlie Lissandrello for the dusty lemonade; apologies to Ryan Dubay, I blame it on the economy; thanks to Ken ``Doctor'' Kotz for your curiosity in my project by always asking if it's working; thanks to Peter, David, Chris and Helen for granting me four square feet in the productivity hub known simply as ``The Pitt''; thanks to Jac, Sarah, and Sammy G. for staying up until 4am on a work night; thanks to Jason Fiering for allowing me to distract his best workers.


Shout out to my funding agency and my employer, the United States Air Force, where, maybe, just once, someone will call me ``Sir'' without adding, ``You're making a scene.''

\cleardoublepage

% The abstractpage environment sets up everything on the page except
% the text itself.  The title and other header material are put at the
% top of the page, and the supervisors are listed at the bottom.  A
% new page is begun both before and after.  Of course, an abstract may
% be more than one page itself.  If you need more control over the
% format of the page, you can use the abstract environment, which puts
% the word "Abstract" at the beginning and single spaces its text.

\begin{abstractpage}
Recent advancements in multilayer, multicellular, genetic logic circuits often rely on manual intervention throughout the computation cycle and orthogonal signals for each chemical ``wire''. These constraints can prevent genetic circuits from scaling. Microfluidic devices can be used to mitigate these constraints. However, continuous-flow microfluidics are largely designed through artisanal processes involving hand-drawing features and accomplishing design rule checks visually: processes that are also inextensible. Additionally, continuous-flow microfluidic routing is only a consideration during chip design and, once built, the routing structure becomes ``frozen in silicon,'' or for many microfluidic chips ``frozen in polydimethylsiloxane (PDMS)''; any changes to fluid routing often require an entirely new device and control infrastructure. The cost of fabricating and controlling a new device is high in terms of time and money; attempts to reduce one cost measure are, generally, paid through increases in the other. 

This work has three main thrusts: to create a microfluidic fabrication framework, called MakerFluidics, that lowers the barrier to entry for designing and fabricating microfluidics in a manner amenable to automation (Chapter \ref{chapter:mf}); to prove this methodology can design, fabricate, and control complex and novel microfluidic devices (Chapter \ref{chapter:xposer}); and to demonstrate the methodology can be used to solve biologically-relevant problems (Chapter \ref{chapter:acoust}).

Utilizing accessible technologies, rapid prototyping, and scalable design practices, the MakerFluidics framework has demonstrated its ability to design, fabricate and control novel, complex and scalable microfludic devices. This was proven through the development of a reconfigurable, continuous-flow routing fabric driven by a modular, scalable primitive called a transposer. In addition to creating complex microfluidic networks, MakerFluidics was deployed in support of cutting-edge, application-focused research at the Charles Stark Draper Laboratory. Informed by a design of experiments approach using the parametric rapid prototyping capabilities made possible by MakerFluidics, a plastic blood--bacteria separation device was optimized, demonstrating that the new device geometry can separate bacteria from blood while operating at 275\% greater flow rate as well as reduce the power requirement by 82\% for equivalent separation performance when compared to the state of the art. 

Ultimately, MakerFluidics demonstrated the ability to design, fabricate, and control complex and practical microfluidic devices while lowering the barrier to entry to continuous-flow microfluidics, thus democratizing cutting edge technology beyond a handful of well-resourced and specialized labs. 


%This research addresses each of the above issues in turn by applying lessons learned from the evolution of electronic computation, ultimately resulting in a microfluidic fabrication and control paradigm called MakerFluidics. Automation via computer aided design (CAD) and computer aided manufacturing (CAM) is stressed through the use of algorithmic frameworks, accessible technologies, rapid prototyping, and scalable design practices. These efforts are focused towards lowering the barrier to entry for continuous-flow microfluidics, thus democratizing the utilization of cutting edge technology beyond a handful of well-resourced and specialized labs. This can only be accomplished by viewing accessibility as a constraint to development.

\end{abstractpage}
\cleardoublepage

% Now you can include a preface. Again, use something like
% \newpage\section*{Preface} followed by your text

% Table of contents comes after preface
\tableofcontents
\cleardoublepage

% If you do not have tables, comment out the following lines
\newpage
\listoftables
\cleardoublepage

% If you have figures, uncomment the following line
\newpage
\listoffigures
\cleardoublepage

% List of Abbrevs is NOT optional (Martha Wellman likes all abbrevs listed)
\chapter*{List of Abbreviations}
\begin{center}
  \begin{tabular}{lll}
    \hspace*{2em} & \hspace*{1in} & \hspace*{4.5in} \\
    ASIC & \dotfill & Application Specific Integrated Circuit \\
    CAD  & \dotfill & Computer-Aided Design \\
    CAM  & \dotfill & Computer-Aided Manufacturing \\
    FWHM & \dotfill & Full-Width at Half Maximum \\
    IDAST &\dotfill & Identification and Antibiotic Susceptibility Test\\
    mLSI & \dotfill & Microfluidic Large Scale Integration \\
    PCB  & \dotfill & Printed Circuit Board \\
    PDMS & \dotfill & Polydimethylsiloxane \\
    $\mathbb{R}^{2}$  & \dotfill & the Real plane \\
    UV	 & \dotfill & Ultra-Violet \\
    VLSI & \dotfill & Very Large Scale Integration \\
  \end{tabular}
\end{center}
\cleardoublepage

% END OF THE PRELIMINARY PAGES

\newpage
\endofprelim
