Recent advancements in multilayer, multicellular, genetic logic circuits often rely on manual intervention throughout the computation cycle and orthogonal signals for each chemical ``wire''. These constraints can prevent genetic circuits from scaling. Microfluidic devices can be used to mitigate these constraints. However, continuous-flow microfluidics are largely designed through artisanal processes involving hand-drawing features and accomplishing design rule checks visually: processes that are also inextensible. Additionally, continuous-flow microfluidic routing is only a consideration during chip design and, once built, the routing structure becomes ``frozen in silicon,'' or for many microfluidic chips ``frozen in polydimethylsiloxane (PDMS)''; any changes to fluid routing often require an entirely new device and control infrastructure. The cost of fabricating and controlling a new device is high in terms of time and money; attempts to reduce one cost measure are, generally, paid through increases in the other. 

This work has three main thrusts: to create a microfluidic fabrication framework, called MakerFluidics, that lowers the barrier to entry for designing and fabricating microfluidics in a manner amenable to automation (Chapter \ref{chapter:mf}); to prove this methodology can design, fabricate, and control complex and novel microfluidic devices (Chapter \ref{chapter:xposer}); and to demonstrate the methodology can be used to solve biologically-relevant problems (Chapter \ref{chapter:acoust}).

Utilizing accessible technologies, rapid prototyping, and scalable design practices, the MakerFluidics framework has demonstrated its ability to design, fabricate and control novel, complex and scalable microfludic devices. This was proven through the development of a reconfigurable, continuous-flow routing fabric driven by a modular, scalable primitive called a transposer. In addition to creating complex microfluidic networks, MakerFluidics was deployed in support of cutting-edge, application-focused research at the Charles Stark Draper Laboratory. Informed by a design of experiments approach using the parametric rapid prototyping capabilities made possible by MakerFluidics, a plastic blood--bacteria separation device was optimized, demonstrating that the new device geometry can separate bacteria from blood while operating at 275\% greater flow rate as well as reduce the power requirement by 82\% for equivalent separation performance when compared to the state of the art. 

Ultimately, MakerFluidics demonstrated the ability to design, fabricate, and control complex and practical microfluidic devices while lowering the barrier to entry to continuous-flow microfluidics, thus democratizing cutting edge technology beyond a handful of well-resourced and specialized labs. 


%This research addresses each of the above issues in turn by applying lessons learned from the evolution of electronic computation, ultimately resulting in a microfluidic fabrication and control paradigm called MakerFluidics. Automation via computer aided design (CAD) and computer aided manufacturing (CAM) is stressed through the use of algorithmic frameworks, accessible technologies, rapid prototyping, and scalable design practices. These efforts are focused towards lowering the barrier to entry for continuous-flow microfluidics, thus democratizing the utilization of cutting edge technology beyond a handful of well-resourced and specialized labs. This can only be accomplished by viewing accessibility as a constraint to development.
