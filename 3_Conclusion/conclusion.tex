\chapter{Conclusions and Future Work}
\label{chapter:Conclusions}
\thispagestyle{myheadings}

% set this to the location of the figures for this chapter. it may
% also want to be ../Figures/2_Body/ or something. make sure that
% it has a trailing directory separator (i.e., '/')!
\graphicspath{{3_Conclusion/Figures/}}

The primary goal of this thesis was to lower the barrier of entry into the field of continuous flow microfluidics. This was achieved by developing a rapid, cost-effective, design-to-fabrication framework for microfluidics using accessible CAM technologies such as desktop CNC mills and 3D printers. Having demonstrated this new rapid prototyping framework is capable of designing, fabricating, and controlling complex and experimentally relevant devices, the next logical step would be to bring this research back to its genesis. 

This work was motivated by the desire to orchestrate large networks of novel synthetic biological systems. I presented a framework capable of designing a device that separates smaller genetic logic components into disparate reaction chambers and connecting these chambers using microchannels, through which biological signals can be sent. My framework provides a mechanism to describe and execute temporally specified valve control sequences, which can control the flow of biological signaling information such that larger functionality can be achieved. 

Additionally, the state of the genetic circuits can be monitored using 
