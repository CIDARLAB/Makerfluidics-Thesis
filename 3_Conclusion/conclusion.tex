\chapter{Conclusions and Future Work}
\label{chapter:Conclusions}
\thispagestyle{myheadings}

% set this to the location of the figures for this chapter. it may
% also want to be ../Figures/2_Body/ or something. make sure that
% it has a trailing directory separator (i.e., '/')!
\graphicspath{{3_Conclusion/Figures/}}

One major goal of this thesis was to lower the barrier of entry into the field of continuous flow microfluidics. This was achieved by developing a rapid, cost-effective, CAD-friendly, design-to-fabrication framework for microfluidics using accessible CAM technologies such as desktop CNC mills and 3D printers. Proving the efficacy of this framework involved demonstrating its capacity to solve relevant and complex experimental problems. The scalability of the framework was proven by its ability to design, fabricate, and control a complex network of novel microfluidic primitives, as shown in Chapter \ref{chapter:xposer}. Its experimental relevance was established in Chapter \ref{chapter:acoust} by optimizing an acoustofluidic blood--bacteria separation device in a thermoplastic substrate --- something that has never been done before.

Having demonstrated this new rapid prototyping framework is capable of designing, fabricating, and controlling complex and experimentally relevant devices, the next logical step would be to bring this research back to its genesis. This work was motivated by the desire to orchestrate large networks of novel synthetic biological systems. I presented a framework capable of designing a device that separates smaller genetic logic components into disparate reaction chambers and connecting these chambers using microchannels, through which biological signals could be sent. My framework provides a mechanism to describe and execute temporally specified valve control sequences, which can control the flow of biological signaling information such that larger functionality can be achieved. 

Additionally, the state of the genetic circuits can be monitored using on-chip diagnostics such as the integration of sensors and optical reporting areas. This on-chip feedback can be integrated into the temporal valve-control specification, thus necessitating an extension to the Biostream language beyond ``wait'' and ``set'' statements to include some element of temporal logic such as ``wait until''.

All of these extensions should be fully integrated into the larger CAD workflow shown in Figure \ref{fig:fullflow}. This would require forgoing the use of OpenSCAD and replacing it with domain-specific CAD tools such as Fluigi Place and Route \cite{hu2014} using MIcrofluidic NeTlist (MINT) \cite{krishna2017iwbda} descriptions. Smaller devices could be designed using a microfluidic-specific drawing tool such as 3D$\mu$F \cite{lippai2017iwbda}. All designs could be informed by a design-automation platform driven by fluid mechanics \cite{ali2017iwbda}, which would provide aspects of automation to device parameter estimation. The design tools listed are compatible with MakerFluidics in that they can export two-dimensional graphics files (i.e., SVG files), however these tools could be extended to output designs in three dimensions. It must be noted that all of the preceding design tools are made relevant by the ability to fabricate the designs they create; this relevance is boosted by the accessibility MakerFluidics brings to continuous-flow microfluidics.


My results show that low-cost microfluidic fabrication techniques are possible and relevant. The integration of continuous-flow microfluidics into day-to-day life in a typical wet lab has the ability to change the way experimentalists look at the nature of their work. Freed from the mundane and costly burdens of device design, fabrication, and running protocols, experimentalists could spend more time on data analysis and experimental documentation --- effectively boosting the quality and quantity of research. MakerFluidics and the capabilities it enables represent an important step towards this integration.
